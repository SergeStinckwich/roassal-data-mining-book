% -*- mode: latex; -*-
\ifx\wholebook\relax\else

	% Lulu:
	\documentclass[a4paper,10pt,twoside]{book}
	\usepackage[
		papersize={6.13in,9.21in},
		hmargin={.75in,.75in},
		vmargin={.75in,1in},
		ignoreheadfoot
	]{geometry}
	\input{../support/latex/common.tex}
	% \input{../support/latex/commonLuaTex.tex}
	\setboolean{lulu}{true}
% --------------------------------------------
% A4:
%	\documentclass[a4paper,11pt,twoside]{book}
%	\input{../support/latex/common.tex}
%	\usepackage{a4wide}
% --------------------------------------------    
\graphicspath{
	{figures/}
	{../figures/}
}

\begin{document}
\fi
\sloppy

\chapter{Introduction}
This book has the goal of proving the efficiency and simplicity of Roassal concerning modelisations for data mining. It exposes most commons techniques of modelisation for data mining, each of them performed using Roassal and explained. 
\section{Data Mining}
\begin{todo}

petite introduction au data mining ?
\end{todo}
\section{Roassal}
\begin{todo}

intro Roassal, how to install it
\end{todo}
\section{A Principle of builders}
The idea of a Rossal builder is to make the creation of a modelisation faster, easier in term of code understanding but still flexible as much as possible.
Based on that principle and given a list of builders, Roassal cover all the typical modelisations.

Histogtam: RTGrapher combined with a RTHistogramSet   

\begin{todo}

list of graphs and builders associated
\end{todo}

From now on, it will be assumed that the reader has a Pharo image opened with a recent version of Roassal installed in order to try the examples. In any case we will try to make this examples as clear as possible, and will always be executable in a Pharo workspace.   



\ifx\wholebook\relax\else
   \bibliographystyle{jurabib}
   \nobibliography{scg}\end{document}
\fi
